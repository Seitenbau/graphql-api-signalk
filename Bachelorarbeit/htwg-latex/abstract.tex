\chapter*{Abstract}
\setheader{Abstract}

Beim Aufbau einer Web-API fällt die Entscheidung für die Technologie in den meisten Fällen auf REST. In den letzten Jahren wurden aber immer mehr Alternativen entwickelt. Dazu gehört auch GraphQL. Als Unternehmen stellt sich jetzt die Frage, wann sich der Einsatz von GraphQL lohnt. Ziel dieser Arbeit ist es, dafür Entscheidungskriterien aufzustellen und zu untersuchen.
\\
Um dieses Ziel zu erreichen, werden erst einige Grundlagen erforscht. Dabei sollen zuerst die wichtigsten Technologien dieser Arbeit erläutert werden. Anschließend wird festgelegt, welche Technologien genutzt werden sollen, um Aspekte der APIs zu untersuchen. Dabei wird auch erläutert, welche Aspekte analysiert werden sollen. Daraufhin wird in einer Kombination aus Codebeispielen und Literaturinformationen die Analyse durchgeführt. Zuletzt werden Use-Cases aufgestellt und im Kontext der Ergebnisse der Analyse evaluiert.
\\
Die Arbeit zeigt, dass die flexiblere Datenübertagungen bei GraphQL viele Vorteile ermöglicht. Es wurde aber auch ersichtlich, dass für sehr einfache APIs der Einsatz von REST zu empfehlen ist.

