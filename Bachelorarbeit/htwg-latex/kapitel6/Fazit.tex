\chapter{Fazit}

Ziel dieser Arbeit war es festzustellen, wann sich der Einsatz von GraphQL gegenüber \ac{REST} für eine Web-API lohnt. Zu diesem Ziel wurden die Aspekte Verbreitung, Implementierung, Dokumentation, Validierung und Antwortzeit untersucht.\\
Bei der Verbreitung konnten kaum Unterschiede zwischen GraphQL und \ac{REST} nachgewiesen werden. Beide können in eigentlich allen sinnvollen Programmiersprachen umgesetzt werden. Dasselbe gilt auch für die Validierung. Zwar hat hier GraphQL mit dem Typesystem einen leichten Vorteil, wenn man aber kompliziertere Validierungen durchführen muss, müssen beide Technologien auf Bibliotheken zurückgreifen. Diese Aspekte haben folglich kaum bis gar keinen Einfluss auf die Entscheidung.\\
Betrachtet man die grundlegende Implementierung, hat \ac{REST} einen leichten Vorteil, welcher vor allem bei einfachen APIs zum Tragen kommt. Das liegt daran, dass bei GraphQL sowohl die Anfragen als auch das Schema in einer eigenen Sprache formuliert werden müssen. Die Anfragen sind dabei auch bei erfahrenen Entwicklern wesentlich komplizierter als die meisten REST-Anfragen.\\
Sobald aber die Komplexität ansteigt, kann GraphQL durch eine einfachere Dokumentation und eine verbesserte Antwortzeit punkten. Die Dokumentation wird dabei von GraphQL automatisch übernommen. Bei \ac{REST} muss für die gleiche Funktionalität auf Bibliotheken zurückgegriffen werden. Die gute Antwortzeit ist ein Resultat des flexiblen Datenzugriffs von GraphQL. Dadurch können die bei \ac{REST} verbreiteten Probleme des Over- und Underfetching vermieden werden. Dieser flexible Datenzugriff ermöglicht auch eine unabhängigere Entwicklung von Front- und Backend, was ein weiterer Vorteil für GraphQL ist. Der Aspekt des Caching, welcher bei GraphQL häufig kritisch gesehen wird, wurde in dieser Arbeit als relativ äquivalent zwischen beiden Technologien erkannt. Zwar funktioniert das automatische Browsercaching bei \ac{REST} sofort, das Gleiche kann bei GraphQL aber auch mit einem GET-Endpunkt erreicht werden. Nur in Ausnahmefällen versagt diese Lösung.\\
Zusammengefasst lässt sich sagen, dass für die untersuchten Aspekte einfache APIs eher \ac{REST} wählen sollten, während kompliziertere APIs mehr auf GraphQL setzten sollten.\\
Wie aussagekräftig diese Empfehlungen sind, hängt aber noch von vielen weiteren Faktoren ab. Einige wichtige Aspekte konnten bspw. im Rahmen dieser Arbeit nicht mehr behandelt werden. Dazu gehören unter anderem, wie gut sich größere binäre Dateien, wie Bilder mit \ac{REST} und GraphQL übertragen lassen. Außerdem müssen vorhandene Bibliotheken noch genauer betrachtet werden, da ihre Features möglicherweise eine Lösung für entscheidende Probleme bieten. Zuletzt sind noch einige Use-Cases nicht betrachtet worden, da hier noch weitere Arbeit vonnöten ist. Dazu gehören vor allem Softwarelösungen mit einer verteilten Architektur wie Microservices.\\